Planning is the act of devising a goal-leading course of action -- a plan.
Planning methods using boolean satisfiability (SAT) have proven effective in the past.
Problematically, these SAT encodings view the plan as a structure-less object and ignore a lot of the inherent structures in planning problems.
As a result, the generated SAT formulae are unnecessarily complex and are thus harder to solve then actually necessary.
Over the past decade, several new techniques for exposing these structures have been developed (e.g., operator counting, AND/OR landmarks, Merge\&Shrink abstractions), but have not yet found their way into SAT-based planning.

I will explore how SAT-based planning can benefit from knowledge about such structures -- opening new research directions in SAT-based planning.



Traditionally, SAT-based planning views the plan as a sequence of identical time-steps without any prior known structure.
This is problematic as planning problems often implicitly enforce action structures or patterns to occur in plans.
I will study how to extract implicit action structures that are useful for SAT-based planning (e.g., using landmarks and operator counting) and how to exploit them.
I will also investigate how a more expressive planning formalism (Factored Transition Systems) can be used to better represent the action model in a SAT formula.

My recent work showed that SAT-based planning is well suited to handle large planning problems, whose actions and predicates cannot be ground instantiated (grounded).
This work treats the whole problem as if it cannot be ground instantiated -- and thus ignores again implicit problem structures.
In most cases, only parts of the problem are too large to be grounded, while others can be.
I will introduce partial grounding, which determines those parts of a planning problem that can and should be grounded.
I will propose ways to handle partially grounded planning problems in SAT-based planning in order to improve its efficiency.


Lastly, I will investigate the connection of SAT-based planning and machine learning (ML).
SAT-based techniques yield provably correct and provably optimal plans, while this is hard or impossible for ML-based ones.
Using ML parameters can be determined during search that solely influence runtime, but not correctness.
Thus, one gains efficiency without loosing theoretical guarantees.

