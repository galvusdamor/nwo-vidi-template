%  This LaTeX-file can be used to submit an application to the NWO Open Competition Domain Science M.
%  It comes without any guarantee that it works on all platforms. In case you run into problems, you can
%  contact us by e-mail: enw-M@nwo.nl, but searching the internet might prove to be a faster solution.

% Start the LaTeX-document
\documentclass[a4paper,9.5pt,fleqn]{application}


% Load the style sheet
\usepackage{application}

% Set spacing
\linespread{1.08333}

%%%%%%%%%%%%%%%%%%%%%%%%%%%%%%%%%%
%%% IMPORTANT FOR BIBLIOGRAPHY %%%
%%%%%%%%%%%%%%%%%%%%%%%%%%%%%%%%%%
% Comment these lines when NOT using bibtex (but thebibliography)
\usepackage[
backend=biber,
maxbibnames=99,
maxnames=99
]{biblatex}
\usepackage{tikz}
\usepackage{bibentry}
%\renewcommand{\bibsection}{}
%%%%%%%%%%%%%%%%%%%%%%%%%%%%%%%%%%
\addbibresource{application.bib}


\newcommand{\makeauthorbold}[1]{%
  \DeclareNameFormat{author}{%
    \ifthenelse{\value{listcount}=1}
    {%
      {\expandafter\ifstrequal\expandafter{\namepartfamily}{#1}{\mkbibbold{\namepartfamily\addcomma\addspace \namepartgiveni}}{\namepartfamily\addcomma\addspace \namepartgiveni}}
      %
    }{\ifnumless{\value{listcount}}{\value{liststop}}
        {\expandafter\ifstrequal\expandafter{\namepartfamily}{#1}{\mkbibbold{\addcomma\addspace \namepartfamily\addcomma\addspace \namepartgiveni}}{\addcomma\addspace \namepartfamily\addcomma\addspace \namepartgiveni}}
        {\expandafter\ifstrequal\expandafter{\namepartfamily}{#1}{\mkbibbold{\addcomma\addspace \namepartfamily\addcomma\addspace \namepartgiveni\addcomma\isdot}}{\addcomma\addspace \namepartfamily\addcomma\addspace \namepartgiveni\addcomma\isdot}}%
      }
    \ifthenelse{\value{listcount}<\value{liststop}}
    {\addcomma\space}{}
  }
}
\makeauthorbold{Applicant}



\DeclareFieldFormat{doi}{\newline DOI: \href{https://doi.org/#1}{\nolinkurl{#1}}}
\DeclareFieldFormat{url}{\newline URL: \href{#1}{\nolinkurl{#1}}}


\usepackage{lipsum} 



\newcommand{\todo}[1]{\textcolor{red}{\textbf{TODO:} #1}}
\newcommand{\maybehide}[1]{#1}

\usepackage{titlesec}
\titlespacing*{\paragraph}{0pt}{4.2ex}{1ex}
\titleformat{\paragraph}[runin]{\bfseries}{}{0pt}{\underline}

% Set the type of application (M-1, M-2 or M-Invest), changes the header
\ifdefined\haveFirstLit{}
\lhead{Grant application preproposal form 2023 NWO Talent Programme – Vidi scheme}
\else
\lhead{}
\fi

% Start the main text
\begin{document}
%  How to fill out this application form?
%
%  IMPORTANT: When writing your proposal, please take into account that it will be assessed by both expert referees
%  as well as more broadly composed cluster committees and domain wide assessment committees.
%
% This application form consists of two parts. Part A is devoted to the scientific proposal, including abstract, summary and a
% justification for the project budget.Part B contains the name and affiliation of the main applicant and additional
% information which is not immediately necessary for the scientific proposal but contains administrative information aiding
% the assessment procedure. Only Part A will be assessed by the referees and the cluster and domain wide assessment
% committees. Please adhere to the following rules when filling out this application form:
% * remove the examples and comments before converting the application to PDF and submitting it;
% * use the font and size as specified in the .sty-file and do not change the margins (2 cm in either direction);
% * each of the two Parts A and B should start on a new page;
% * no budget table should be included in this application form; please use the separately provided spreadsheet to 
%    construct your budget table and upload the pdf of it together with this application form;
% * the basic details of the proposal (section A.1, incl. abstract and summary) are limited to one page;
% * the scientific proposal (section A.2) is limited to six pages, which includes figures and tables, but excludes the 
%    list of literature references.
% * it is not allowed to include hyperlinks to a personal website, to a group website or to similar information which, in fact,
%    extends the page limit set;
% * it is not allowed to include ‘CV information’ such as prizes, (prestigious) grants, et cetera;
% * NWO signed the San Francisco Declaration on Research Assessment (DORA). DORA aims to call a halt to the
%    irresponsible use of bibliometric indicators in assessing research and researchers (such as the H-index, Journal Impact Factor and
%    citations). It is a global initiative for all research disciplines (for more information about this, see https://sfdora.org/). NWO
%    implements its principles in all instruments. Therefore it is not allowed to mention H-indices, Journal  Impact Factors, overall citations
%    or similar bibliometric indicators in M-applications. It is allowed to mention the number of citations of a single output item that is
%    relevant to the proposed research;
% * it is not allowed to include letters of support or documents other than required.
 
% What to submit to NWO through ISAAC?
% Upload as a separate pdf:
% * Application form
% * Budget table
% * If applicable: letter request for preferential treatment signed by applicant
% * If applicable: employer's statement when 
%       ~ preferential treatment is being applied for (including extension clause)
%       ~ the contract of the applicant (tenure track or upcoming retirement) does not cover the runtime of the project:
%           guarantee adequate supervision of the to be appointed personnel


%%%%%%%%%%%%%%%%%

%\bibliographystyle{unsrt}



%\section{Literature references}
%  The list of references should be relevant to the research proposal, cited in the texts of section A.2, and only include
%  'open literature'. It should not include internal documents (such as master theses) nor be a mere list of publications of
%  the main applicant. This section does not count toward the limitation of six pages.

% EXAMPLE: Using bibtex
%\bibliography{application}

\ifdefined\haveScience
\part{Institution and field of research}
\section{NWO Domain}
Science (ENW)

\section{Main field of research}
XX.XX.XX Some field
\fi


%%%%%%%%%%%%%%%%%%% profile section
\ifdefined\haveProfile
\newpage
\ifdefined\haveOutput{} 
\else
\renewcommand{\fullcite}[1]{\qquad}
\renewcommand{\maybehide}[1]{\qquad}
\fi

%\renewcommand{\maybehide}[1]{#1}



\maybehide{\part{Evidence Based Curriculum Vitae}}

\maybehide{\section{Academic Profile}}

\maybehide{\subsection{General Academic Profile}}

\lipsum[1-6]



\maybehide{\subsection{Leadership and Mentorship}}

\lipsum[1-3]


\fi

\ifdefined\haveFirstLit
\vspace{1cm}Total word count \thesection{} Academic Profile: \input{wordProfile.tex} of 1200\\
\fi


%%%%%%%%%%%%%%%%%%% output section
\ifdefined\haveOutput
\newpage
%\nobibliography*

\ifdefined\haveProfile{} 
\else
\renewcommand{\fullcite}[1]{\qquad}
\renewcommand{\maybehide}[1]{\qquad}
\fi



\newcounter{indicatorCounter}
\setcounter{indicatorCounter}{0}
\newlength{\heightA}
\settoheight{\heightA}{A}

\newcommand{\openacc}[1]{\maybehide{Open Access: #1 
\ifnum\pdfstrcmp{Yes}{#1}=0
  \includegraphics[height=\heightA]{Open_Access_logo.png}
\fi
}}
\newcommand{\pubtype}[1]{\maybehide{Type: #1}}
\newcommand{\indicator}[2]{\maybehide{Output Indicator \refstepcounter{indicatorCounter} (#1): #2}}



\maybehide{\section{Key output}}

\begin{enumerate}[label={\maybehide{(\arabic*)}}]
%%%% Theory results!
\item \openacc{Yes}\\
	\fullcite{CITATION}\\
	\pubtype{Conference Paper}\\
	\indicator{Citation total number}{999.999}\\
	\maybehide{Motivation:} This is the best paper.

\end{enumerate}

\ifdefined\haveFirstLit{} 
%\bibliography{application}
%\nobibliography{application}
\else
%\bibliography{application}
%\nobibliography{application}
\fi

\fi

\ifdefined\haveFirstLit
\vspace{1cm}Total word count \thesection{} Key Output: \input{wordOutput.tex} of 400-700\\
\fi

\ifdefined\haveAbstract
\ifdefined\haveOutput 
\else
\renewcommand{\maybehide}[1]{\qquad}
\fi

\maybehide{\section{Research idea}}
\maybehide{\textbf{Key Words: }}

\lipsum[1-2]

\fi

\ifdefined\haveFirstLit
\vspace{1cm}Total word count \thesection{} Research Idea: \input{wordAbstract.tex} of 150\\
\fi

\ifdefined\haveFirstLit

\newpage
\part{Administrative details}
\section{Personal Details}
Title, initial, surname: Dr.\ T.\ Applicant\\
Preferred language of correspondence: English


\section{Master's Degree}
\begin{tabular}{lllllllll}
University/College of Higher Education& University of Foo\\
Main subject& 
\end{tabular}

\section{Doctorate}
\begin{tabular}{lllllllll}
University/College of Higher Education:& Bar University\\
Starting date (dd/mm/yy)& 01/01/1999\\
Date of PhD award (dd/mm/yy)&   01/01/2019\\
Supervisor(s) (``Promotor(es)'')& Prof.\ Dr.\ T.\ Supervisor\\
Thesis title& 
\end{tabular}

\section{Prospective Host Institution}

\begin{tabular}{lllllllll}
Host institution& University of Amsterdam\\
Research group& Institute for Logic, Language, and Computation
\end{tabular}

\section{Work Experience since Completing Your (First) PhD}

\begin{tabular}{lllllllll}
Position & Periode & FTE & Position Type & Institution\\ \hline
PostDoc & 01/2020 - 02/2022 & 1.0 & fixed term & University of Gak\\
Assistant Professor (UD) & 03/2022 - 08/2023 & 1.0 & fixed term & University of Amsterdam\\
Assistant Professor (UD) & since 09/2023 & 1.0 & permanent & University of Amsterdam
\end{tabular}


\section{Net. Academic Research Time}
\begin{tabular}{lllllllll}
Experience& Number of Months\\
Research activities & 26.6 (= 18.2 + 8.4)\\
Education & 18.3 (= 7.8 + 10.5)\\
Leave & 0 \\
Management Tasks & 2.1 \\
\end{tabular}

\paragraph{Percentage of research, teaching and management per position}
\begin{enumerate}
	\item 01/2020 - 02/2022 (26 months, 1.0 fte): 70\% research, 30\% teaching
	\item 03/2022 - 11/2023 (21 months, 1.0 fte): 40\% research, 50\% teaching, 10\% management
\end{enumerate}

\paragraph{Calculation months of research}
\begin{enumerate}
	\item 26 months * 1.0 fte * 70\% research = 18.2 months\\ 
	      26 months * 1.0 fte * 30\% teaching = 7.8 months
	\item 21 months * 1.0 fte * 40\% research = 8.4 months\\
	      21 months * 1.0 fte * 50\% teaching = 10.5 months\\
	      21 months * 1.0 fte * 10\% management = 2.1 months
\end{enumerate}


\newpage

\section*{Statements by the applicant}
\subsection*{Use of extension clause }
If you make use of the extension clause, (only) add the date of the e-mail you have received from \texttt{talent@nwo.nl} with a confirmation that your extension was granted. An extension is only necessary if you exceed the year limit on the reference date. 

\noindent Do you make use of the extension clause: No\\
If yes, the extension was confirmed on:


\subsection*{By submitting this form I declare that:}
By submitting this form, I endorse the code of conduct for laboratory animals and the code of conduct for biosecurity/possibility for dual use of the expected results and will act accordingly, if applicable.

\noindent $\boxtimes$ I have completed this form truthfully.\\
$\boxtimes$ I have submitted the completed and signed embedding guarantee.\\
$\boxtimes$ I declare that I satisfy the nationally and internationally accepted standards for scientific conduct as stated in the Netherlands Code of Conduct for Research Integrity 2018.\\[1cm]

\noindent Initial(s) and surname(s):\\
Place:\\
Date:\\


\fi

\nocite{*}


% And we're done
\end{document}
